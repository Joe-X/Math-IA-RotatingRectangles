% ======================================================================== %
%   Rotating Rectangle Task (Math HL Practice IA Project)
%       Joe Xu
%       Y12D
%       2016-11-26
%       Ms. Many
% ======================================================================== %

\documentclass{article}

% Title Information
\title  {Rotating Rectangles Task}
\date   {November 20, 2016}
\author {Joe Xu}


% Required Packages %

% For Symbols
\usepackage{amsmath}
\usepackage{amssymb}
\usepackage{gensymb}
% For Diagrams
\usepackage{graphicx}
% For Bibliography
\usepackage[backend=bibtex,style=verbose-trad2,maxbibnames=99]{biblatex}
\bibliography{RRect}


% ------------------------------------------------------------------------ %
% Document Structure                                                       %
% ------------------------------------------------------------------------ %
%  Title
%  Abstract 
%  Table of Contents
%  Prerequisites
%  Approaching The Problem
%  Calculating Coordinates of a 20-degrees-rotated Rectangle
%  Conclusion 1
%  Futher Investigation
%  Appendix
%  Works Cited
% ------------------------------------------------------------------------ %


% Document
\begin{document}
    \pagenumbering{gobble}
    \maketitle
    \newpage
    
    \section*{Abstract}
        This paper is created to answer the questions given on the "Rotating Rectangles" task sheet. The questions are focused on  Trigonometry and Geometry.
        \newline\newline
        This paper is structured in accordance to the task sheet. That is, the first part of the paper focuses on calculating coordinates of an image of a rectangle rotated by given degrees. The second part of this paper focuses on the more general nature of the trigonometric functions sine and cosine.
        \newline\newline
        The approach to calculate the coordinates of the image of a rectangle rotated by given degrees (20\degree \ used for illustration of the question) is to use a combination of trigonometry (sine, cosine and tangent) properties and identities, as well as graphical methods such as calculating distance between two points on a 2D plane.
        \newline\newline
        GeoGebra\autocite{GEOGEBRA:1} is used for the diagrams, while TeXstudio\autocite{TEXSTUDIO:1} is used to produce this formatted paper.
    \newpage
    
    \tableofcontents
    \newpage
    
    \pagenumbering{arabic}
        
    \section{Prerequisites}
        \subsection{Trigonometric Properties and Identities}
        Trigonometric properties such as the Pythagorean theorem and trigonometric identities such as the Pythagorean identities and compound angle identities are utilized throughout this paper for proofs and for calculations.
        \paragraph{Trigonometric Properties} These are basic properties involving right-angled triangles as well as the unit circle.
            \subparagraph{Labels on a Right-Angled Triangle} These labels will be used by the following properties and identities.
            \begin{figure}[h!]
                \includegraphics[width=\linewidth]{RRectPreq1.png}
                \caption{Labeled Right-Angled Triangle}
                \label{fig:tri1}
            \end{figure}\newline
            Figure \ref{fig:tri1} shows a labeled triangle, with vertices $A$, $B$ and $C$, and edges $a$, $b$, and $c$. Edge $a$ is the adjacent leg of the right-angle triangle; it forms the angle $\theta$ and a right-angle with edges $b$ and $c$ respectively. Edge $a$ is the opposite leg of the triangle; it is the edge opposite to the angle $\theta$. Edge $c$ is the hypotenuse of the triangle; it is the longest edge of the triangle.
            \subparagraph{Pythagorean Theorem}\label{PYTH:1} The Pythagorean theorem states that for a right-angled Triangle (see Figure \ref{fig:tri1}), the square of the adjacent leg plus the opposite leg equals to the square of the hypotenuse.
            \begin{align*}
                adjacent^2 + opposite^2 &= hypotenuse^2 \\
                a^2 + b^2 &= c^2
            \end{align*}
            \subparagraph{Basic Trigonometric Functions}\label{TRIG:1} Sine, Cosine and Tangent, as well as Co-secant(csc) and Secant(sec). The angle $\theta$ is used here for the right-angled Triangle, but these functions can take bigger or smaller angles as well (with exceptions for tangent).
            \begin{align*}
                \cos{\theta} &= \frac{adjacent}{hypotenuse} \left(= \frac{a}{c}\right) \\
                \sin{\theta} &= \frac{opposite}{hypotenuse} \left(= \frac{b}{c}\right) \\
                \tan{\theta} &= \frac{\sin{\theta}}{\cos{\theta}} = \frac{opposite}{adjacent} \left(= \frac{b}{a}\right) \\
                \sec{\theta} &= \frac{1}{\cos{\theta}} \\
                \csc{\theta} &= \frac{1}{\sin{\theta}} \\
            \end{align*}
            \subparagraph{The Unit Circle}\label{UC:1} The unit circle can help to give the exact values of special angles, 30\degree, 45\degree, 65\degree, etc. The radius of the circle is one ($r=1$). The radius of the circle is the hypotenuse, which can form a right-angled triangle with the x-axis, by the vertical extension from the intersection of the radius with the circle's circumference to the x-axis.
            \begin{figure}[h!]
                \includegraphics[width=\linewidth]{unit-circle-degrees.png}
                \caption{The Unit Circle}
                \label{fig:uc1}
            \end{figure}\newpage
            This diagram is from OnlineMathLearning\autocite{UNIT_CIRCLE:1}.
        \paragraph{Trigonometric Identities} These include the Pythagorean identities and the compound angle identities.
            \subparagraph{Pythagorean Identities}
            \begin{align}
                \cos{\theta}^2 + \sin{\theta}^2 &\equiv 1 \\
                1 + \tan{\theta}^2 &\equiv \sec{\theta} \\
                \cot{\theta} + 1 &\equiv \csc{\theta}
            \end{align}
            \subparagraph{Compound Angle Identities}
            \begin{align}
                \sin{(\alpha \pm \beta)} &= \sin{\alpha}\cos{\beta} \pm \cos{\alpha}\sin{\beta} \\
                \cos{(\alpha \pm \beta)} &= \cos{\alpha}\cos{\beta} \mp \sin{\alpha}\sin{\beta}
            \end{align}
            %%%%%%%%%%
            \newpage
        \subsection{Graphical Properties and Calculations} This includes the distance formula.
        \paragraph{Calculating Distance Between Two Points} This formula calculates the distance (length) between two points, $A$ and $B$ (or the length of segment, $AB$). Let point $A$ and $B$ be $(X_1, Y_1)$ and $(X_2, Y_2)$ respectively, where X and Y denotes the coordinates of $A$ and $B$.
        \begin{equation}
            AB_{\mathrm{DISTANCE}} = \sqrt{(X_2 - X_1)^2 + (Y_2 - Y_1)^2}
        \end{equation}
        
    %%%%%%%%
    \newpage
    \section{Approaching The Problem} As per the questions from the task sheet, I am to calculate the coordinates of the image of a rectangle rotated by given degrees. There exists multiple methods to calculate the coordinates, but to get results as accurate and as exact as possible, I have decided to use trigonometrical functions in conjunction with the unit circle (modifying the radius) to calculate the coordinates. To do so, the investigation is based on a case study of the image of a rectangle $OABC$ rotated $20\degree$ from the x-axis, anti-clockwise, about the origin, $O$; the length and width of the rectangle is 4 units and 3 units respectively.
    \begin{figure}[h!]
        \includegraphics[width=\linewidth]{RRect1.png}
        \caption{Image of Rectangle OABC Rotated 20$\degree$}
        \label{fig:rrect1}
    \end{figure} \newline
    Figure \ref{fig:rrect1} shows the rectangle $\mathrm{OABC}$ rotated $20\degree$ from the x-axis about the origin $O$, giving the image $\mathrm{OA_1B_1C_1}$.
    \\\\
    I will calculate the coordinates of the image in terms of sines and cosines in order to deduce general expressions for calculating the coordinates.
    \newpage
        \subsection{Calculating $A_1$} In order to calculate $A_1$, I isolated the segment $\mathrm{OA_1}$ from the image to form a right-angled triangle with the x-axis. This is demonstrated in the following diagrams:
        \begin{figure}[h!]
            \includegraphics[width=\linewidth]{RRect2.png}
            \caption{Right-angled Triangle Formed By $OA_1$}
            \label{fig:rrect2}
        \end{figure} \newline
        Then, I treated the hypotenuse (the length of the rectangle $OA_1 = 4$) as the radius of a unit circle multiplied by the length $OA_1$, i.e. a circle with radius $r = 4$.
        \begin{figure}[h!]
            \includegraphics[width=\linewidth]{RRect3.png}
            \caption{Right-angled Triangle Formed By $OA_1$}
            \label{fig:rrect3}
        \end{figure} \\
        
    \newpage
    
\end{document}
